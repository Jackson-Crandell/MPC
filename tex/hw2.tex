\documentclass[11pt]{homework}
\usepackage[margin=1in]{geometry}
\usepackage{float}
\fancyheadoffset{0cm}

\newcommand{\hwname}{Luis Pimentel ~~~~~~~~~~~~~~ Jackson Crandell}
\newcommand{\professor}{Professor Evangelos Theodorou}
\newcommand{\hwemail}{lpimentel3@gatech.edu ~~~ jackcrandell@gatech.edu}
\newcommand{\hwtype}{Homework}
\newcommand{\hwnum}{2}
\newcommand{\hwclass}{AE 4803 Robotics and Autonomy}
\newcommand{\hwdate}{November 2, 2020}

\renewcommand{\vec}[1]{\ensuremath{\boldsymbol{#1}}}
\addtolength{\jot}{0.5em}
\allowdisplaybreaks

\usepackage[colorlinks=false]{hyperref}
\usepackage[ruled,vlined]{algorithm2e}

\usepackage{minted}
\usepackage{ dsfont }

\begin{document}
\maketitle

%%%%%%%%%%%%%%%%%%%%%%%%%%%%%%%%%%%%%%%%%%%%%%%%%%%%%%%%%%%%%%%%%%%%%%%%%%%%%%%%%%%%%%% Part 1 
\renewcommand{\questiontype}{Part}
\question
	See MATLAB code implementation. 

	To run this simulation run the Inverted\_Pendulum/main.m file.

%%%%%%%%%%%%%%%%%%%%%%%%%%%%%%%%%%%%%%%%%%%%%%%%%%%%%%%%%%%%%%%%%%%%%%%%%%%%%%%%%%%%%%% Part 1 
\renewcommand{\questiontype}{Part}
\question
	See MATLAB code implementation. 

	To run this simulation run the CartPole/main.m file.
	
%%%%%%%%%%%%%%%%%%%%%%%%%%%%%%%%%%%%%%%%%%%%%%%%%%%%%%%%%%%%%%%%%%%%%%%%%%%%%%%%%%%%%%% Part 3 
\setcounter{questionCounter}{2}
\renewcommand{\questiontype}{Part}
\question
\begin{arabicparts}
%\setcounter{questionCounter}{2}




\questionpart
See MATLAB code implementation.

To run this simulation run the Inverted\_Pendulum/main\_robustness.m file. \\

	In this problem we are asked to test the robustness of our MPC-DDP policy on a nomial model of the system used in the DDP feedback optimization and a real model used to propagate forward the dynamics. 
	
	To test robustness we create the real model by perturbing the nominal model by different levels of uncertainty denoted by $\sigma$. The parameters of the inverted pendulum are perturbed as follows. 
	
	\begin{align*}
		& g_{real} = abs(g_{nominal} + (0.5)^{\sigma}g_{nominal}(2rand - 1)\sigma^2) \\
		& m_{real} = abs(m_{nominal} + m_{nominal}(2rand - 1)\sigma) \\
		& b_{real} = abs(b_{nominal} + b_{nominal}(2rand - 1)\sigma) \\
		& l_{real} = abs(l_{nominal} + l_{nominal}(2rand - 1)\sigma) \\
		& I_{real} = abs(I_{nominal} + I_{nominal}(2rand - 1)\sigma) \\
	\end{align*}
	
	In this update rand is a random number between 0 and 1. As the inverted pendulum can tolerate high levels of $\sigma$ the parameter gravity is scaled down much more as to model Earth's gravity closer. The absolute value is taken as these parameters cannot be negative. 
	
	In our test we increase $\sigma$, compute a new real model with this uncertainty level, and rerun MPC-DDP until it cannot converge anymore. 
	
	This process is detailed through the following algorithm:\\
	
	\begin{algorithm}[H]
		\SetAlgoLined
 		Initialize: $\vec{F_{nominal}}, \vec{F_{real}}$: nominal and real dynamics model\;
 		Initialize: $\sigma$: = 0 uncertainty level, $\delta_{\sigma}$\;
 		
 		\While{}{
 			$\vec{F_{real}} $ = update\_real\_model($\sigma, \vec{F_{nominal}}$) \;
  			\While{}{
  				$\vec{u} = $ fnDDP(\vec{F_{nominal}},\vec{x_{now}})\;
  				$\vec{x}_{next} = $ fnDynamics($\vec{F_{real}}, \vec{x_{now}}, \vec{u}$)\;
  				Repeat until task completion. Break if not converged and greater than $i_{max}$ iterations
  			}
  			$\sigma$ = $\sigma$ + $\delta_{\sigma}$ \;
  			Break if last MPC-DDP not converged and greater than $i_{max}$ iterations
 		}
 		\caption{MPC-DDP Robustness Test Against Perturbations in Nominal Model}
	\end{algorithm}

	The following plot shows the results of MPC-DDP after this robustness test is run for several levels of uncertainty:
	
		\begin{figure}[H]
			\centering
			\includegraphics[scale=0.20]{robustness_results_pendulum.png}
			\caption{Robustness Test results for Inverted Pendulum.}
		\end{figure}
		
		The following plot shows the results of the perturbation in parameters of the model used to create the real model, for several levels of uncertainty:
	
		\begin{figure}[H]
			\centering
			\includegraphics[scale=0.20]{robustness_results_params_pendulum.png}
			\caption{Robustness Test results for Inverted Pendulum.}
		\end{figure}
		\begin{figure}[H]
			\centering
			\includegraphics[scale=0.25]{robustness_results_params_pendulum2.png}
			\caption{Robustness Test results for Inverted Pendulum.}
		\end{figure}

		\newpage
		
\questionpart
See MATLAB code implementation.

To run this simulation run the CartPole/main\_robustness.m file. \\

		The robustness test described above was executed for the Cart Pole problem. For this we use the following real model update rule. 
		
		\begin{align*}
			& g_{real} = abs(g_{nominal} + (0.5)g_{nominal}(2rand - 1)\sigma^2) \\
			& m\_c_{real} = abs(m\_p_{nominal} + m\_c_{nominal}(2rand - 1)\sigma) \\
			& m\_p_{real} = abs(m\_c_{nominal} + m\_p_{nominal}(2rand - 1)\sigma) \\
			& l_{real} = abs(l_{nominal} + l_{nominal}(2rand - 1)\sigma) \\
		\end{align*}
		
		
		The following are the results. 
	
		\begin{figure}[H]
			\centering
			\includegraphics[scale=0.165]{robustness_results_cart_pole.png}
			\caption{Robustness Test results for Cart Pole.}
		\end{figure}
		
		The following plot shows the results of the perturbation in parameters of the model used to create the real model, for several levels of uncertainty:
	
		\begin{figure}[H]
			\centering
			\includegraphics[scale=0.17]{robustness_results_params_cart_pole.png}
			\caption{Robustness Test results for Cart Pole.}
		\end{figure}
		\begin{figure}[H]
			\centering
			\includegraphics[scale=0.17]{robustness_results_params_cart_pole2.png}
			\caption{Robustness Test results for Cart Pole.}
		\end{figure}

\end{arabicparts}

	







\end{document}
